\begin{frame}
\frametitle{Beispielanwendung}
\begin{itemize}
	\item Die Beispielanwendung besteht aus drei Java-Microservices
	\pause
	\item \textbf{Producer} erstellt Arbeitsaufträge und sendet diese an eine asynchrone Warteschlange. 
	\begin{itemize}
		\item<.-> \textbf{WAIT}: Kommando bewirkt einen Wartezyklus von 5~Sekunden im Worker
		\item<.-> \textbf{NOOP}: Kommando bewirkt eine Ausführung eines Nullbefehls
		\item<.-> \textbf{FIBONACCI}: Kommando bewirkt die Bereichnung einer Fibonacci-Zahl im Worker
		\item<.-> \textbf{KILL}: Kommando bewirkt, dass der Worker beendet wird.  
	\end{itemize}
	\pause
	\item \textbf{Worker} nimmt die Kommandos aus der Warteschlange und bearbeitet diese
	\begin{itemize}
		\item Abgearbeitete Kommandos werden in die Warteschlage \textbf{results} eingestellt 
	\end{itemize}
	\item \textbf{Writer} Der Prozess nimmt die Ergebnisse aus der Warteschlange \textbf{results} und persistiert diese.
\end{itemize}
\end{frame}
