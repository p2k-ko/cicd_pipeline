\begin{frame}
\frametitle{CI/CD}

\begin{definition}
	\textbf{CI} : engl. \textit{Continious Integration}\\
	\textbf{CD} : engl. \textit{Continious Delivery} oder \textit{Continious Deploy} 
\end{definition}

\pause
Was soll kontinuierlich integriert werden?
\pause
\textbf{Softwareanpassungen!}\\
\pause 
Wo liegt der Unterschied zwischen \textit{Delivery} und \textit{Deploy}?\\
\pause
Beim Continious Delivery findet die Aktivierung im Produktivsystem manuell statt. 

\end{frame}

\begin{frame}
	\frametitle{Verfügbare CI/CD-Lösungen}
	\begin{itemize}
		\item Jenkins
		\item Gitlab 
		\item Travis-CI
		\item TeamCity
		\item CircleCI
		\item (uvm.)
	\end{itemize}
	Jede dieser Anwendungen besitze Stärken und Schwächen die im jeweiligen Projektkontext bewertet werden müssen. \\
	
	Die nachfolgenden Beispiele wurden mit GitLab und dessen CI/CD-Möglichkeiten erstellt.
\end{frame}